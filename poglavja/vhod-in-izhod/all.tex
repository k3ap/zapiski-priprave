\naslov{Osnovna struktura programa}

Programi za svoje delovanje potrebujejo način za komunikacijo z uporabnikom.
Kompleksnejši programi v ta namen uporabljajo ekran, miško in tipkovnico, pri
tekmovalnem programiranju pa najpogosteje uporabljamo najpreprostejši način za
komunikacijo: pisanje in branje s \emph{standardnega vhoda in izhoda}.
Običajno to pomeni, da se nam ob zagonu programa odpre okno, kamor lahko pišemo
programu in kamor program izpisuje stvari.
Ko želimo, da naš program kaj izpiše, uporabimo \emph{funkcijo} \koda{printf}.
Poglejmo si enostaven primer.

\fkoda{poglavja/vhod-in-izhod/helloworld.cpp}

Funkciji \koda{printf} v dvojnih narekovajih damo besedilo ali števila, ki jih
želimo izpisati.
Na koncu tega besedila napišemo \verb+\n+, ki označuje, da mora program na tem
mestu iti v novo vrstico.
To je pomembno vključiti predvsem, če funkcijo \koda{printf} uporabimo večkrat
zaporedoma, saj bi bilo sicer celotno besedilo izpisano v eni vrstici.

Posvetimo se tudi splošni obliki zgornje kode, saj vsebuje ključne elemente, ki
jih mora vsebovati vsak program.
Prva vrstica, \verb+#include <stdio.h>+, pove programu, da bomo uporabljali
funkcije za vhod in izhod, konkretno \koda{printf} in kasneje \koda{scanf}.
Če te vrstice nebi napisali, bi ob poskusu izvajanja kode sistem javil napako,
in trdil, da funkcije \koda{printf} ne pozna.

Besedilo \koda{int main()} računalniku pove, da bodo sledili zaviti oklepaji (to
so oklepaji, ki izgledajo \{takole\}), znotraj katerih bo glavno telo naše kode.
Zaenkrat bomo vso našo kodo napisali med te zavite oklepaje, ko pa bomo spoznali
sezname in kasneje funkcije, bomo nekaj kode vnesli tudi drugam.
Koda v \koda{main} je organizirana v vrstice, ki se morajo končati s podpičjem
\verb+;+.
Ko se bo program izvedel, se bodo zaporedoma od zgoraj navzdol izvedle vse
vrstice, dokler ne pridemo do zadnje vrstice, ki se mora začeti z ukazom
\koda{return}.
Temu ukazu sledi številka --- ta pove, če se je med izvajanjem programa zgodila
kakšna napaka.
Če je številka enaka $0$, se je program končal brez napak, drugim številkam pa
pravimo \emph{kode napake}.
Te so uporabne predvsem zato, da lahko uporabnik programerju le z eno številko
pove, kakšna napaka se je v programu zgodila.
Mi bomo v prihodnje večinoma pisali programe, katerih uporabniki bomo sami, zato
bomo vedno uporabili kodo $0$.

V program lahko dodamo \emph{komentarje}.
To je besedilo, ki je sicer napisano v kodi programa, a ne vpliva na njegov
potek, ker računalnik komentarjev ne izvede.
Zaradi tega lahko komentarji vsebujejo tudi besedilo v naravnem jeziku
(slovenščini), ki programerjem razlaga pomen kode poleg komentarja.
Na voljo imamo dve vrsti komentarjev:
\begin{itemize}
\item Če na začetku vrstice napišemo dve poševnici, \verb+//+, s tem dobimo
  komentar, ki prikriva besedilo v tej vrstici.
  Če se ta komentar pojavi sredi vrstice, bo prikril vse besedilo od tam naprej
  do konca vrstice.
\item Če v besedilu zapišemo poševnico in zvezdico, \verb+/*+, s tem dobimo
  komentar, ki prikriva vso besedilo do vključno prve pojavitve nasprotnega
  simbola, \verb+*/+.
\end{itemize}

\fkoda{poglavja/vhod-in-izhod/komentarji.cpp}

\naslov{Branje podatkov}

Programu lahko sporočimo različne podatke, program pa mora te podatke nekam
shraniti, preden jih lahko obravnava.
Mestu, kamor podatke shranimo, pravimo \emph{spremenljivka}, saj lahko te
podatke med tekom programa spreminjamo.
Vsem spremenljivkam v programu damo ime, s katerim se na njih sklicujemo, ter
\emph{podatkovni tip}, ki pove, kakšni podatki so v spremenljivki shranjeni
(npr.~besedilo, številka, \ldots).
V spodnjem primeru ustvarimo eno spremenljivko, ki jo imenujemo
\koda{tvoje_ime}, njen tip pa je \enquote{besedilo dolžine največ $50$}.

\fkoda{poglavja/vhod-in-izhod/branje.cpp}

Tudi \koda{scanf} je funkcija, ki ji podamo dva ali več \emph{parametrov}.
Prvi parameter mora vedno biti niz znotraj narekovajev, ki opisuje, kakšnega
tipa so podatki, ki naj jih funkcija prebere.
Ta opis podamo s \emph{formatnikom}, v zgornjem primeru \verb+%s+ računalniku
pove, da bo program prebral eno besedo.
Preostali parametri povedo, v katero spremenljivko naj funkcija shrani prebrane
podatke.
Če imamo v nizu več formatnikov, moramo podati eno spremenljivko za vsak
formatnik.

Do zdaj smo funkciji \koda{printf} podali samo točno določeno besedilo, ki smo
ga želeli izpisati.
Izpisujemo pa lahko tudi spremenljivke, kot smo to naredili v tem zadnjem
primeru.
Znotraj besedila dodamo formatnike na mesta, kjer želimo, da so spremenljivke,
potem pa izven narekovajev naštejemo imena spremenljivk, ki jih želimo izpisati.

V zgornjem primeru smo brali in izpisali niz besedila, kar pa je pravzaprav
zahtevnejše od branja in pisanja števil.
Za delo z nizi potrebujemo kompleksnejše ukaze, ki jih bomo spoznali kasneje,
zato se bomo do nadaljnjega omejili na delo s (celimi) števili.
Tem pripada tip \koda{int} (angl.~\textit{integer}) ter formatnik \verb+%d+,
kakor vidimo v naslednjem primeru.

\fkoda{poglavja/vhod-in-izhod/branje-stevil.cpp}

Pri branju števil imamo le eno dodatno zahtevo kot pri branju nizov --- pred ime
spremenljivke moramo zapisati znak \verb+&+.
Razlog za tem bomo spoznali, ko bomo obravnavali kazalce, za sedaj pa to vzemimo
kot zahtevo postopka.
Pri številih tudi ne povemo direktno največje dolžine, kakor smo to naredili pri
nizih, saj je največja velikost določena že vnaprej.
Tip \koda{int} lahko shrani pozitivna in negativna števila velikosti največ 2
milijardi.

% LocalWords:  formatnikov formatnik formatnikom formatnike
