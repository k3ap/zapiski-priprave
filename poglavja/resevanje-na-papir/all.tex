\naslov{Tekmovanje RTK}

Na šolskem nivoju tekmovanja RTK ter v nekaterih skupinah tekmovanja na državnem
nivoju se programerske naloge rešuje \enquote{na papir.}
To ne pomeni, da moraš rešitve dejansko z roko napisati na list papirja, temveč
le, da oddaje ne bodo avtomatsko ocenjene, temveč jih bo pregledal eden od
popravljavcev, rezultat pa dobiš šele po koncu tekmovanja.
Pišeš lahko bodisi na računalnik, bodisi na list papirja.

Ker tvojega programa ne bo nihče prevedel ali pognal, so sprejemljive tudi
rešitve, ki niso popolnoma sintaktično pravilne --- za napake kot manjkajoče
podpičje na koncu vrstice, napačen formatnik v \koda{printf}, tipkarski škrati
ipd.~se praviloma odšteje le nekaj točk, kljub temu da tak program na
običajnem tekmovanju nebi dosegel nobene točke.
Med drugim to tudi pomeni, da lahko oddajaš delne rešitve.
Če nalogo na primer razdeliš na dva manjša problema, enega od katerih znaš
rešiti, drugega pa ne, lahko zapišeš rešitev prvega podproblema in opišeš, kako
bi jo lahko kombiniral z rešitvijo drugega problema.
Za tako oddajo seveda ne dobiš vseh točk, če pa je rešen del problema dovolj
pomemben, pa ti popravljavci lahko dodelijo delne točke.

Ker programa ne bo pregledal računalnik, pač pa človek, moraš svojo rešitev
napisati tako, da bo bralcu jasno, kaj tvoja rešitev počne.
To med drugim pomeni, da uporabljaš deskriptivna imena spremenljivk, da v
komentarjih programa razložiš, kakšno idejo tvoja rešitev uporablja, ter da na
kratko utemeljiš, kaj naredi posamezen blok kode.
Koda mora biti tudi lepo poravnana, da bo bralec lahko že s pogledom videl, kako
je program strukturiran.

Včasih navodila naloge dopuščajo pisanje rešitev v psevdokodi.
To pomeni, da lahko rešitev zapišeš v besedah, a jo strukturiraš kot da bi bila
koda --- torej tako, da lahko vsakdo, ki je vsaj malo vešč v programiranju, tvoj
zapis enostavno prevede v svoj najljubši programski jezik.
Ko pišeš v psevdokodi, uporabljaš standardne strukturne elemente programov
(\koda{if} stavke, \koda{for} zanke itd.), vendar jih zapišeš v slovenščini, ne
kot kodo.

\naslov{Primeri}

Spodaj je nanizanih nekaj primerov rešitev za naloge iz šolskega tekmovanja
MladiRTK 2024, ki so dostopne na
\href{https://rtk.ijs.si/2024/mladirtk2024-solsko-naloge-resitve-kriterij.pdf}{tej
povezavi}.
Vse naloge so vredne $25$ točk.

\podnaslov{Pribor}

Kot prvi primer si poglejmo program spodaj.
Iz opisa je razvidno, da je tekmovalec prebral nalogo, in da ve, kateri funkciji
se v C++ uporabljata za branje in pisanje iz standardnega vhoda in izhoda,
vendar je zapisana rešitev vredna $0$ točk.
Tekmovalec ni nakazal, kako bi program dejansko napisal, temveč je le v svojih
besedah zapisal besedilo naloge.

\fkoda{poglavja/resevanje-na-papir/pribor-zelo-slabo.cpp}

Primerjajmo to z rešitvijo, ki je vredna vseh $25$ točk.
Ne samo, da spodnja koda pravilno reši problem, to dejstvo je tekmovalec tudi
utemeljil.
Koda je berljiva, poleg tega pa je na vsakem koraku utemeljeno, kakšno vlogo
ima vsak del kode.

\fkoda{poglavja/resevanje-na-papir/pribor-popolno.cpp}

Spodnja rešitev namesto programa poda opis rešitve.
Naloga je zahtevala kodo ali psevdokodo, zato za opisno rešitev odbijemo nekaj
točk, a je iz spodnjega opisa jasno, da tekmovalec razume, kako bi kodo rešitve
dejansko zapisal, zato še vedno prejme delne točke.
Žal pa so tudi v opisu napake, za katere dodatno odbijemo nekaj točk.
Tekmovalec je števila na vhodu shranil v spomin, za kar odbijemo $2$ točki,
kakor za to nalogo priporočajo nasveti za popravljanje, poleg tega pa je
tekmovalec pozabil posebej obravnavati prvo prebrano število, za kar odbijemo
$4$ točke.
Opisan postopek tudi ne omeni, da moramo na koncu še izpisati višino zadnjega
stolpa, izpis namreč zahteva le ob obravnavi prevelike žlice.
Ta rešitev je tako vredna $10$ točk.

\fkoda{poglavja/resevanje-na-papir/pribor-opisno.txt}

Poglejmo si še eno rešitev v psevdokodi.
Če bi spodnje besedilo prepisali v C++, bi bila rešitev pravilna, razen tega, da
bi tekmovalec za primerjavo moral uporabiti \koda{<=} namesto \koda{<}, saj v
navodilih piše, da lahko zložimo dve enako veliki žlici eno nad drugo.
V navodilih za popravljanje je navedeno, da za tako napako odbijemo $5$ točk.
Načeloma bi bila potem na tekmovanju taka rešitev vredna $20$ točk, a bo
popravljavec po lastni presoji lahko dodatno odbil točke zaradi stila pisanja.
Vse spremenljivke v kodi so enočrkovne, prav tako pa noben del rešitve ni
utemeljen.
Če bi imel tekmovalec v kodi napako ali bi kakšen del rešitve napisal z nejasno
oznako, bi lahko zaradi pomanjkanja razumevanja popravljavec taki rešitvi dal
mnogo manj točk, saj nebi bilo razvidno, da tekmovalec nalogo sploh razume.

\fkoda{poglavja/resevanje-na-papir/pribor-psevdokoda.txt}

\podnaslov{Palindromski oklepaji}

Ključna opazka pri reševanju naloge je, da niz ne mora biti hkrati zrcalen in
palindrom.
Če je niz palindrom, mora namreč prvi znak biti enak zadnjemu, če pa je zrcalen,
pa mora biti prvi znak zrcalna slika zadnjega.
Navodila popravljanja za to opazko predvidijo $6$ točk, tudi če rešitev ne reši
naloge.
Nadaljnje reševanje lahko razdelimo na dva neodvisna dela --- preverjanje, ali
je niz palindrom, in preverjanje, ali je niz zrcalen.

Spodnji program preveri le, če je niz palindrom, tekmovalec pa je dodal še
opombo, da je niz lihe dolžine vedno bodisi palindrom bodisi navaden.
Takšna rešitev bi na tekmovanju dobila $13$ točk.

\fkoda{poglavja/resevanje-na-papir/oklepaji-palindrom.cpp}

V spodnji rešitvi tekmovalec pravilno določi, če je niz palindrom in če je
zrcalen, a manjka razmislek o tem, da niz ne more biti hkrati palindrom in
zrcalen, za kar odbijemo nekaj točk.
Poleg tega je tekmovalec pozabil prebrati niz in namesto \texttt{zrcalen izraz}
izpisal le \texttt{zrcalen}, za kar dodatno odbijemo dve točki.
Tekmovalec prejme $20$ točk.

\fkoda{poglavja/resevanje-na-papir/oklepaji-pomankljiv-razmislek.cpp}

% LocalWords:  formatnik printf enočrkovne zrcalnosti
